\documentclass{article}
\usepackage[utf8]{inputenc}
\usepackage[brazil]{babel}
\usepackage{amsmath}
\usepackage{amssymb}
\usepackage{amsthm}
\newtheorem{theorem}{Teorema}

\title{
    INE5451 - Tópicos Especiais em Algoritmos I \\
    Introdução à Criptoanálise \\
    Relatorio II
}
\author{
    Lucas Zarbato - 11100890\\
    Tiago Royer - 12100776
}

\date{3 de Julho de 2015}

\begin{document}

\maketitle

\section{Introdução}

O trabalho a ser entregue consistia de seis questões
relacionadas ao conteúdo aprendido em aula.
As três primeiras eram obrigatórias
e envolviam a implementação do ataque integral ao AES,
implementação do RSA
e o desenvolvimento de um ataque linear a uma SPN.
As últimas três eram eletivas e precisavam ser escolhidas
dentre um leque de 6 opções.
Escolhemos fazer as últimas três questões, relacionadas ao log discreto.

Novamente, fizemos todos os programas em C++.
Nosso código está disponível em \texttt{https://github.com/royertiago/INE5451}.

\end{document}
