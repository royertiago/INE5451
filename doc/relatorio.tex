\documentclass{article}
\usepackage[utf8]{inputenc}
\usepackage[brazil]{babel}
\usepackage{amsmath}
\usepackage{amssymb}
\usepackage{amsthm}
\newtheorem{theorem}{Teorema}

\title{
    INE5451 - Tópicos Especiais em Algoritmos I \\
    Introdução à Criptoanálise \\
    Relatorio I
}
\author{
    Lucas Zarbato - 11100890\\
    Tiago Royer - 12100776
}

\date{14 de Maio de 2015}

\newcommand{\sw}{\mathit{sw}}
\newcommand{\IP}{\mathit{IP}}

\begin{document}

\maketitle

\section{Introdução}

O trabalho a ser entregue consistia de quatro questões
relacionadas ao conteúdo aprendido em aula.
Duas destas questões exigiam implementação
de um algoritmo para quebrar chaves,
e uma pedia que fizéssemos análise de frequência
de uma grande quantidade de textos escritos em alemão.

Por questão de eficiência, fizemos todos os programas em C++.
Nosso código está disponível em \texttt{https://github.com/royertiago/INE5451}.

\section{Questão 1 --- Análise de Frequencia do Alemão}

Montamos um pequeno programa para contar todas as ocorrências de caracteres
e de pares de caracteres em textos escritos em alemão.

A fonte dos textos foi a Wikipedia em alemão.
A Wikipedia disponibiliza apenas os dados em XML;
para simplificar o parsing, pegamos os dados já em texto puro
deste site: \texttt{http://kopiwiki.dsd.sztaki.hu/}.
Utilizamos a biblioteca Boost.Locale para lidar com o ß (eszett)
e com acentuação.

Foram analisadas, no total, $5,8$ bilhões de letras.
As frequências, em ordem desendente, estão disponíveis na tabela \ref{letras}.

\begin{table}[h]
    \centering
    \begin{tabular}{r l}
        Letra & Frequência \\\hline
            e & 0,15123 \\
            n & 0,09098 \\
            i & 0,08026 \\
            r & 0,07657 \\
            a & 0,06792 \\
            s & 0,06757 \\
            t & 0,06202 \\
            d & 0,04642 \\
            u & 0,04333 \\
            h & 0,04083 \\
            l & 0,04074 \\
            o & 0,03540 \\
            c & 0,02967 \\
            g & 0,02867 \\
            m & 0,02707 \\
            b & 0,02201 \\
            f & 0,01693 \\
            k & 0,01642 \\
            w & 0,01417 \\
            p & 0,01222 \\
            z & 0,01120 \\
            v & 0,00996 \\
            j & 0,00364 \\
            y & 0,00291 \\
            x & 0,00120 \\
            q & 0,00052 \\
    \end{tabular}
    \caption{Frequência de aparição das letras no alemão.}
    \label{letras}
\end{table}

Tabelamos também todos pares de letras.
Nós consideramos que o final de uma palavra e o início de outra formam um par.
Como alguns pares quase não aparecem
(jx, qy e qx, por exemplo, aparecem poucos milhares de vezes),
optamos por mostrar apenas os 20 maiores,
na tabela \ref{pares}.
Eles respondem por cerca de 32\% de todos os pares.

\begin{table}[h]
    \centering
    \begin{tabular}{r l}
        Par & Frequência \\\hline
        er & 0,03515 \\
        en & 0,02960 \\
        ch & 0,02285 \\
        de & 0,02015 \\
        ei & 0,01787 \\
        te & 0,01769 \\
        in & 0,01683 \\
        nd & 0,01675 \\
        ie & 0,01392 \\
        st & 0,01256 \\
        ge & 0,01209 \\
        un & 0,01208 \\
        es & 0,01207 \\
        an & 0,01180 \\
        re & 0,01080 \\
        ne & 0,01028 \\
        he & 0,01009 \\
        be & 0,00974 \\
        is & 0,00971 \\
        sc & 0,00900 \\
        se & 0,00865 \\
    \end{tabular}
    \caption{20 pares com maior taxa de aparição no alemão.}
    \label{pares}
\end{table}

\section{Questão 2 --- Decifragem do DES}

\begin{theorem}
    Para decifrar o DES, basta fornecer as subchaves em ordem reversa.
\end{theorem}

\begin{proof}

    (Durante a demonstração, $x$ e $y$
    representarão números de 32 bits.)

    Defina a função $\sw$ como a função de swap:
    $\sw(xy) = yx$.
    Defina a função $D(xy, K)$ como o resultado de uma rodada do DES,
    com $x$ como palavra da esquerda, $y$ a palavra da direita,
    e $K$ a subchave.

    Primeiro, mostraremos que $D(\sw(D(xy, K)), K) = \sw(xy)$.

    Chame de $x^*y^*$ a saída de $D(xy, K)$.
    A partir da descrição em alto nível do DES,
    sabemos que $x^* = y$
    e que $y^* = x \oplus f(y, K)$.
    Isso é só o que precisamos saber a respeito de uma rodada do DES.

    Temos
    \begin{equation*}
        \sw(D(xy, K)) = \sw(x^*y^*) = y^*x^*.
    \end{equation*}
    Aplicando novamente uma rodada DES,
    obtemos na saída o par $x'y'$,
    em que $x' = x^*$
    e
    \begin{equation*}
        y' = y^* \oplus f(x^*, K).
    \end{equation*}

    Agora, expandindo os valores de $x^*$, $x$, $y^*$ e $y$,
    temos $x' = x^* = y$,
    e
    \begin{align*}
        y'  &= y^* \oplus f(x^*, K) \\
            &= (x \oplus f(y, K)) \oplus f(y, K) \\
            &= x.
    \end{align*}
    Portanto,
    \begin{equation}
        D(\sw(D(xy, K)), K) = x'y' = yx = \sw(xy).
        \label{dsw}
    \end{equation}

    De posse desta equação,
    não precisamos mais inspecionar as metades individuais
    e podemos trabalhar com blocos de 64 bits.

    Chamaremos de $t$ o texto em claro, e $T_0 = \IP(t)$
    ($\IP$ é a permutação inicial).

    Dispomos de uma lista de chaves $K_1, \dots, K_{16}$.
    Chamaremos de $T_i$ a saída da $i$-ésima rodada do DES;
    temos $T_i = D(T_{i-1}, K_i)$ para $0 < i \leq 16$.

    Por fim, realizamos um swap, e aplicamos $\IP^{-1}$.
    A saída do algoritmo é $\IP^{-1}(\sw(T_{16}))$.

    Agora, passaremos esta saída novamente pelo algoritmo,
    mas fornecendo as subchaves na ordem reversa.

    Chamaremos de $C_i$ a saída da $i$-ésima rodada
    desta passagem pelo algoritmo.
    $C_0$ é antes de qualquer rodada;
    como apenas aplicamos $\IP$,
    temos $C_0 = \sw(T_{16})$.

    Provaremos por indução que $C_i = \sw(T_{16-i})$.
    O caso base já está feito.

    Assuma para $C_{i-1}$.
    Para calcular $C_i$,
    usamos a mesma fórmula.
    Observe que usaremos a chave $K_{16}$ para obter $C_1$,
    $K_{15}$ para obter $C_2$;
    em geral, $K_{16 - i + 1}$ para obter $C_i$.
    \begin{align*}
        C_i &= D(C_{i-1}, K_{16-i+1}) \\
            &= D(\sw(T_{16-i+1}), K_{16-i+1})   & \text{pelo passo indutivo} \\
            &= D(\sw(D(T_{16-i}, K_{16-i+1})), K_{16-i+1}) &
                \text{pela definição de $T_{16-i+1}$} \\
            &= \sw(T_{16-i}), &\text{pela equação \ref{dsw}}.
    \end{align*}
    Isso conclui a demonstração por indução.

    Portanto, $C_{16}$, a última saída após todas as rodadas,
    é $\sw(T_0)$.
    Após o swap, recuperamos $T_0$.
    Mas $T_0 = \IP(t)$!
    Portanto, ao aplicar $\IP^{-1}$,
    recuperamos o texto original, $t$,
    como a saída do algoritmo.
\end{proof}

\section{Questão 3 --- Máquina Enigma}

Esta questão fornecia um texto cifrado utilizando a máquina Enigma,
e a informação de que o ajuste dos anéis era \texttt{AAA}
(o que é equivalente a não haver os anéis)
e que a plugboard não continha nenhum cabo.

Entretanto, não sabemos quais os rotores utilizados
nem seus posicionamentos.
O objetivo do exercício era que implementássemos uma Enigma
e descobrissemos a chave, por força bruta.

Existem $26*26*26$ possíveis posições dos rotores
e $5*4*3$ diferentes ordens de rotores,
resultando em pouco mais de um milhão de chaves diferentes.
Claramente não podemos inspecioná-las manualmente.
Nossa abordagem foi de utilizar o índice de coincidência
após cada tentativa de decifragem.

Em essência, o algoritmo é:
itere sobre cada escolha de rotores,
e itere sobre cada posição dos rotores;
passe a mensagem na Enigma e calcule o índice de coincidência;
imprima a chave caso o índice fique acima de um certo limiar.

Experimentalmente, nós impomos o ``limiar de aceitação'' em $0.05$.
Como o texto é grande (cerca de 400 caracteres),
os índices de coincidência para cada candidato a texto claro
raramente passavam de $0.04$.
Os índices de coincidência para linguagens naturais
ficam em torno de $0.07$,
então escolher o limiar em $0.05$ pareceu um bom compromisso entre
não poder descartar a chave real
nem ``deixar passar muito lixo'' --- isto é, falsos positivos.

Este valor funcionou corretamente.
De todo o espaço de chaves possível,
apenas uma apresentava índice de coincidência superior a $0.05$:
rotores \texttt{III-IV-V}, posições \texttt{UFS}.
O texto em claro foi extraído de um texto de Edsger Dijkstra.

\subsection{Otimizações}

Como este era o problema que mais exigia tempo de CPU,
dedicamos um tempo adicional otimizando sua implementação.
A fim de minimizar a quantidade de multiplicações e divisões,
nós não utilizamos diretamente a expressão matemática
que gera a permutação realizada pela Enigma.
Nós nos baseamos em sua descrição física para fazer a implementação.

Usamos a ideia de girar a entrada,
e não o rotor,
para simular a passagem do sinal pelo rotor;
pré-computamos as permutações inversas de cada rotor;
e, dentro da máquina,
sempre trabalhamos com valores no intervalo $[0, 25]$,
portanto, a saída de uma permutação pode ser usada como índice
para a próxima permutação.
Mesmo a computação do índice de coincidência
é realizada sobre valores em $[0, 25]$.

Dessa forma, toda a Enigma é computada usando apenas somas,
acesso a vetores, e a operação $\mod 26$.
Além disso, os vetores indexam \texttt{signed char}s;
são 7 vetores de 26 bytes no total
(dois para cada rotor e mais um para o refletor).
Mesmo com as informações adicionais
(como posição atual e localização do entalhe),
a Enigma inteira cabe em 256 bytes,
tornando nossa implementação bastante amigável à cache do processador.

Compilando com \texttt{gcc 4.9.2} com nível de otimização \texttt{O3},
num processador Intel Core i5 2.3 GHz,
em apenas 30 segundos,
todas as 1 milhão de chaves são analisadas%
\footnote{
    Em teoria, bastaria analisar apenas até encontrar a chave certa.
    Entretanto, para propósitos de benchmark,
    fizemos o programa continuar testando chaves
    mesmo após encontrar uma chave válida.

    No nosso caso, cerca de metade das chaves tem de ser analisadas
    antes de encontrar a ``verdadeira''.
}.

Nós aplicamos mais uma otimização após isso:
remover aqueles $\mod 26$.
Criamos uma função \texttt z que mapeia os inteiros
para valores em $\mathbb{Z}_{26}$.
Esta função só é chamada com somas ou subtrações
de valores entre $0$ e $25$,
portanto, seu argumento está sempre no intervalo $[-26, 2*26-1]$.
Ou seja: ela pode ser substituida por uma tabela de lookup.
Fazer isso triplicou a velocidade do programa;
agora, todo o espaço de chaves é analisado em 10 segundos.

E, por último, nós adicionamos duas heurísticas
que permitem descartar a chave prematuramente.
A primeira, mais conservadora,
parte do princípio de que,
no início das palavras,
muito provavelmente deve aparecer alguma vogal%
\footnote{
    Uma rápida inspeção em \texttt{/usr/share/dict/words}
    sugere que aproximadamente 3\% das palavras
    começam com três não-vogais.
}
Nós analisamos as três primeiras letras do candidato a texto em claro.
Caso não haja vogal alguma nestas letras,
descartamos a chave imediatamente.

Esta heurística dobrou a velocidade de processamento,
o que é compatível com o fato de a probabilidade de não haver vogal
em três letras escolhidas aleatoriamente ser
\begin{equation}
    \frac{21}{26} * \frac{21}{26} * \frac{21}{26} \approx 0.5
\end{equation}

A segunda heurística é mais questionável.
Ela é baseada nas seis letras mais frequentes no inglês,
E, T, A, O, I, N.
Estas seis letras respondem por aproximadamente 51\%
de todas as letras na lingua inglesa.
Portanto, nas 20 letras iniciais,
podemos esperar cerca de 10 letras pertencentes a este conjunto.
Nós decidimos arbitrariamente descartar
qualquer candidato cujas 20 primeiras letras apresentem
menos de 8 letras neste conjunto.

Esta heurística deu resultados melhores.
Dos candidatos que passam no primeiro teste,
empiricamente descobrimos que cerca de 80\% não passam no segundo,
acelerando a computação em 5 vezes.

Paramos de otimizar (e procurar heurísticas) por aqui.
Neste momento, conseguimos quebrar o texto
de qualquer um dos 10 grupos em pouco menos de 1 segundo.

\section{Problema 4 - Criptoanálise do DES-3}

O objetivo deste problema era meramente implementar
o algoritmo de criptoanálise diferencial para o DES com três rodadas,
ingorando a permutação e o swap final.

Não há muito o que comentar deste problema,
pois ele se resumia a implementar o algoritmo já pronto.

Nossa chave (de 56 bits) ficou neste formato:

\texttt{?01001000110?00010100?111?010?10110100?10?0010?000001010}

Os oito bits restantes foram obtidos via força bruta.
Nossa chave é \texttt{0x2331A25215E34038}.

É interessante notar que,
apesar de nos ter sido fornecidos cinco pares de textos,
quaisquer quatro daqueles textos seriam suficientes;
mas a alguns subconjuntos com três pares de textos
não determina unicamente os bits correspondentes da chave.

\end{document}
